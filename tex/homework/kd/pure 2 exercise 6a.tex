\documentclass{../../style}

\begin{document}
	\begin{center}
		\texttt{Pure Book 2, Exercise 6A}

		\section*{Question 1}
		\text{Without using your calculator, write down the sign of the following trigometric ratios.}
		\begin{subequations}
			\begin{dmath}
				\sec(300\degree) = \reciprocal{\cos(300\degree)} = \reciprocal{\cos(60\degree)} \\
				\text{$\cos(\theta) > 0$ in the first quadrant $\therefore \sec(300\degree) > 0$}
			\end{dmath}

			\begin{dmath}
				\csc(190\degree) = \reciprocal{\sin(190\degree)} = \reciprocal{\sin(-10\degree)} = -\reciprocal{\sin(10\degree)} \\
				\text{$\sin(\theta) > 0 $ in the first quadrant $\therefore \csc(190\degree) < 0$}
			\end{dmath}

			\begin{dmath}
				\cot(110\degree) = \reciprocal{\tan(110\degree)} = \reciprocal{-\tan(70\degree)} = -\reciprocal{\tan(70\degree)} \\
				\text{$\tan(\theta) > 0 $ in the first quadrant $\therefore \cot(110\degree) < 0 $}
			\end{dmath}

			\begin{dmath}
				\cot(200\degree) = \reciprocal{\tan(200\degree)} = \reciprocal{\tan(20\degree)} \\
				\text{$\tan(\theta) > 0 $ in the first quadrant $\therefore \cot(200\degree) > 0 $}
			\end{dmath}

			\begin{dmath}
				\sec(95\degree) = \reciprocal{\cos(95\degree)} \\
				\text{$\cos(\theta) < 0 $ in the second quadrant $\therefore \sec(95\degree) < 0 $}
			\end{dmath}
		\end{subequations}

		\section*{Question 2}
		\text{Use your calculator to find, to 3 significant figures, the values of:}
		\begin{subequations}
			\begin{dmath}
				\sec(100\degree) = \reciprocal{\cos(100\degree)} = -5.76 \sig{3}
			\end{dmath}

			\begin{dmath}
				\csc(260\degree) = \reciprocal{\sin(260\degree)} = -1.02 \sig{3}
			\end{dmath}

			\begin{dmath}
				\csc(280\degree) = \reciprocal{\sin(280\degree)} = -1.02 \sig{3}
			\end{dmath}

			\begin{dmath}
				\cot(550\degree) = \reciprocal{\tan{550\degree}} = 5.67 \sig{3}
			\end{dmath}

			\begin{dmath}
				\cot \frac{4}{3}\pi = \reciprocal{\tan \frac{4}{3}\pi} = 0.577 \sig{3}
			\end{dmath}

			\begin{dmath}
				\sec(2.4\rad) = \reciprocal{\cos(2.4\rad)} = -1.36 \sig{3}
			\end{dmath}

			\begin{dmath}
				\csc \frac{11}{10}\pi = \reciprocal{\sin \frac{11}{10}\pi} = -3.24 \sig{3}
			\end{dmath}

			\begin{dmath}
				\sec(6\rad) = \reciprocal{\cos(6\rad)} = 1.04 \sig{3}
			\end{dmath}
		\end{subequations}

		\section*{Question 3}
		\text{Find the exact values (in surd form where appropriate) of the following:}
		\begin{subequations}
			\begin{dmath}
				\csc(90\degree) = \reciprocal{\sin(90\degree)} = 1
			\end{dmath}

			\begin{dmath}
				\cot(135\degree) = \reciprocal{\tan(135\degree)} = \reciprocal{-\tan(45\degree)} = -1
			\end{dmath}

			\begin{dmath}
				\sec(180\degree) = \reciprocal{\cos{180\degree}} = -1
			\end{dmath}

			\begin{dmath}
				\sec(240\degree) = \reciprocal{\cos(240\degree)} = \reciprocal{-\cos(60\degree)} = \reciprocal{-\reciprocal{2}} = -2
			\end{dmath}

			\begin{dmath}
				\csc(300\degree) = \reciprocal{\sin(300\degree)} = \reciprocal{-\sin(60\degree)} = \reciprocal{-\frac{\sqrt{3}}{2}} = -\frac{2}{\sqrt{3}} = -\frac{2\sqrt{3}}{3}
			\end{dmath}

			\begin{dmath}
				\cot(-45\degree) = \reciprocal{\tan(-45\degree)} = \reciprocal{-\tan(45\degree)} = -1
			\end{dmath}

			\begin{dmath}
				\sec(60\degree) = \reciprocal{\cos(60\degree)} = \reciprocal{\reciprocal{2}} = 2
			\end{dmath}

			\begin{dmath}
				\csc(-210\degree) = \reciprocal{\sin(-210\degree)} = \reciprocal{\sin(30\degree)} = \reciprocal{\reciprocal{2}} = 2
			\end{dmath}

			\begin{dmath}
				\sec(255\degree) = \reciprocal{\cos(255\degree)} = \reciprocal{-\cos(45\degree)} = \reciprocal{-\reciprocal{\sqrt{2}}} = -\sqrt{2}
			\end{dmath}

			\begin{dmath}
				\cot \frac{4}{3}\pi = \reciprocal{\tan \frac{4}{3}\pi} = \reciprocal{\tan \frac{\pi}{3}} = \reciprocal{\sqrt{3}} = \frac{\sqrt{3}}{3}
			\end{dmath}

			\begin{dmath}
				\sec \frac{11}{6}\pi = \reciprocal{\cos \frac{11}{6}\pi} = \reciprocal{\cos \frac{\pi}{6}} = \reciprocal{\frac{\sqrt{3}}{2}} = \frac{2}{\sqrt{3}} = \frac{2\sqrt{3}}{3}
			\end{dmath}

			\begin{dmath}
				\csc(-\frac{3}{4}\pi) = \reciprocal{\sin(-\frac{3}{4}\pi)} = \reciprocal{-\sin \frac{\pi}{4}} = \reciprocal{-\reciprocal{\sqrt{2}}} = -\sqrt{2}
			\end{dmath}
		\end{subequations}

		\section*{Question 4}
		\text{Prove that $\csc(\pi - x) \equiv \csc x$.}
		\begin{dmath}
			\csc(\pi - x) = \reciprocal{\sin(\pi - x)} = \reciprocal{\sin(\pi)\cos(x) - \cos(\pi)\sin(x)} = \reciprocal{\sin x} = \csc x
		\end{dmath}

		\section*{Question 5}
		\text{Show that $\cot(30\degree)\sec(30\degree) = 2$.}
		\begin{dmath}
			\frac{\cos(30\degree)}{\sin(30\degree)} \times \reciprocal{\cos(30\degree)} = \reciprocal{\sin(30\degree)} = \reciprocal{\reciprocal{2}} = 2
		\end{dmath}

		\section*{Question 6}
		\text{Show that $\csc \frac{2}{3}\pi + \sec \frac{2}{3}\pi = a + b\sqrt{3}$ where $a$ and $b$ are real numbers to be found.}
		\begin{dmath}
			\csc \frac{2}{3}\pi + \sec \frac{2}{3}\pi = \reciprocal{\sin \frac{2}{3}\pi} + \reciprocal{\cos \frac{2}{3}\pi} = \reciprocal{\sin \frac{\pi}{3}} + \reciprocal{-\cos \frac{\pi}{3}} = \reciprocal{\frac{\sqrt{3}}{2}} + \reciprocal{-\reciprocal{2}} = -2 + \frac{2}{3} \sqrt{3}
		\end{dmath}
	\end{center}
\end{document}
