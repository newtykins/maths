\documentclass{../../main}

\rhead{J. Smith}

\begin{document}
	\title{Pure Book 2, Exercise 6A}

	% Question 1 
	
	\question{Without using your calculator, write down the sign of the following trigometric ratios.}
	
	\begin{arr}[a]
		\sec(300\deg) = \inverse{\cos(300\deg)} = \inverse{\cos(60\deg)} \\
		\text{$\cos(\theta) > 0$ in the first quadrant $\therefore \sec(300\deg) > 0$}
	\end{arr}

	\begin{arr}[b]
		\csc(190\deg) = \inverse{\sin(190\deg)} = \inverse{\sin(-10\deg)} = -\inverse{\sin(10\deg)} \\
		\text{$\sin(\theta) > 0 $ in the first quadrant $\therefore \csc(190\deg) < 0$}
	\end{arr}

	\begin{arr}[c]
		\cot(110\deg) = \inverse{\tan(110\deg)} = \inverse{-\tan(70\deg)} = -\inverse{\tan(70\deg)} \\
		\text{$\tan(\theta) > 0 $ in the first quadrant $\therefore \cot(110\deg) < 0 $}
	\end{arr}

	\begin{arr}[d]
		\cot(200\deg) = \inverse{\tan(200\deg)} = \inverse{\tan(20\deg)} \\
		\text{$\tan(\theta) > 0 $ in the first quadrant $\therefore \cot(200\deg) > 0 $}
	\end{arr}

	\begin{arr}[e]
		\sec(95\deg) = \inverse{\cos(95\deg)} \\
			\text{$\cos(\theta) < 0 $ in the second quadrant $\therefore \sec(95\deg) < 0 $}
	\end{arr}

	% Question 2 
	
	\question{Use your calculator to find, to 3 significant figures, the values of:}
	
	\begin{arr}[a]
		\sec(100\deg) = \inverse{\cos(100\deg)} = -5.76 \sig{3}
	\end{arr}

	\begin{arr}[b]
		\csc(260\deg) = \inverse{\sin(260\deg)} = -1.02 \sig{3}
	\end{arr}

	\begin{arr}[c]
		\csc(280\deg) = \inverse{\sin(280\deg)} = -1.02 \sig{3}
	\end{arr}

	\begin{arr}[d]
		\cot(550\deg) = \inverse{\tan{550\deg}} = 5.67 \sig{3}
	\end{arr}

	\begin{arr}[e]
		\cot \frac{4}{3}\pi = \inverse{\tan \frac{4}{3}\pi} = 0.577 \sig{3}
	\end{arr}

	\begin{arr}[f]
		\sec(2.4\rad) = \inverse{\cos(2.4\rad)} = -1.36 \sig{3}
	\end{arr}

	\begin{arr}[g]
		\csc \frac{11}{10}\pi = \inverse{\sin \frac{11}{10}\pi} = -3.24 \sig{3}
	\end{arr}

	\begin{arr}[h]
		\sec(6\rad) = \inverse{\cos(6\rad)} = 1.04 \sig{3}
	\end{arr}

	% Question 3 
	
	\question{Find the exact values (in surd form where appropriate) of the following:}
	
	\begin{arr}[a]
		\csc(90\deg) = \inverse{\sin(90\deg)} = 1
	\end{arr}

	\begin{arr}[b]
		\cot(135\deg) = \inverse{\tan(135\deg)} = \inverse{-\tan(45\deg)} = -1
	\end{arr}

	\begin{arr}[c]
		\sec(180\deg) = \inverse{\cos{180\deg}} = -1
	\end{arr}

	\begin{arr}[d]
		\sec(240\deg) = \inverse{\cos(240\deg)} = \inverse{-\cos(60\deg)} = \inverse{-\inverse{2}} = -2
	\end{arr}

	\begin{arr}[e]
		\csc(300\deg) = \inverse{\sin(300\deg)} = \inverse{-\sin(60\deg)} = \inverse{-\frac{\sqrt{3}}{2}} = -\frac{2}{\sqrt{3}} = -\frac{2\sqrt{3}}{3}
	\end{arr}

	\begin{arr}[f]
		\cot(-45\deg) = \inverse{\tan(-45\deg)} = \inverse{-\tan(45\deg)} = -1
	\end{arr}

	\begin{arr}[g]
		\sec(60\deg) = \inverse{\cos(60\deg)} = \inverse{\inverse{2}} = 2
	\end{arr}

	\begin{arr}[h]
		\csc(-210\deg) = \inverse{\sin(-210\deg)} = \inverse{\sin(30\deg)} = \inverse{\inverse{2}} = 2
	\end{arr}

	\begin{arr}[i]
		\sec(255\deg) = \inverse{\cos(255\deg)} = \inverse{-\cos(45\deg)} = \inverse{-\inverse{\sqrt{2}}} = -\sqrt{2}
	\end{arr}

	\begin{arr}[j]
		\cot \frac{4}{3}\pi = \inverse{\tan \frac{4}{3}\pi} = \inverse{\tan \frac{\pi}{3}} = \inverse{\sqrt{3}} = \frac{\sqrt{3}}{3}
	\end{arr}

	\begin{arr}[k]
		\sec \frac{11}{6}\pi = \inverse{\cos \frac{11}{6}\pi} = \inverse{\cos \frac{\pi}{6}} = \inverse{\frac{\sqrt{3}}{2}} = \frac{2}{\sqrt{3}} = \frac{2\sqrt{3}}{3}
	\end{arr}

	\begin{arr}[l]
		\csc(-\frac{3}{4}\pi) = \inverse{\sin(-\frac{3}{4}\pi)} = \inverse{-\sin \frac{\pi}{4}} = \inverse{-\inverse{\sqrt{2}}} = -\sqrt{2}
	\end{arr}

	% Question 4
	
	\question{Prove that $\csc(\pi - x) \equiv \csc x$.}
	
	\begin{arr}
		\csc(\pi - x) = \inverse{\sin(\pi - x)} = \inverse{\sin(\pi)\cos(x) - \cos(\pi)\sin(x)} = \inverse{\sin x} = \csc x
	\end{arr}

	% Question 5
	
	\question{Show that $\cot(30\deg)\sec(30\deg) = 2$.}
	
	\begin{arr}
		\frac{\cos(30\deg)}{\sin(30\deg)} \times \inverse{\cos(30\deg)} = \inverse{\sin(30\deg)} = \inverse{\inverse{2}} = 2
	\end{arr}

	% Question 6
	
	\question{Show that $\csc \frac{2}{3}\pi + \sec \frac{2}{3}\pi = a + b\sqrt{3}$ where $a$ and $b$ are real numbers to be found.}
	
	\begin{arr}
		\csc \frac{2}{3}\pi + \sec \frac{2}{3}\pi = \inverse{\sin \frac{2}{3}\pi} + \inverse{\cos \frac{2}{3}\pi} = \inverse{\sin \frac{\pi}{3}} + \inverse{-\cos \frac{\pi}{3}} = \inverse{\frac{\sqrt{3}}{2}} + \inverse{-\inverse{2}} = -2 + \frac{2}{3} \sqrt{3}
	\end{arr}
\end{document}
