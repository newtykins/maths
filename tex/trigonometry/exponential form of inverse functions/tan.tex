\documentclass{../../style}

\begin{document}
\eulerformula

\begin{gather*}
	\therefore \sin(\theta) = \frac{e^{i\theta} - e^{-i\theta}}{2i} \\
	\cos(\theta) = \frac{e^{i\theta} + e^{-i\theta}}{2} \\
	\tan(\theta) = \frac{\sin(\theta)}{\cos{\theta}} = -\frac{i(-1 + e^{2i\theta})}{1 + e^{2i\theta}}
\end{gather*}

\begin{gather*}
	\text{let} \tan(\theta) = x \\
	x(1 + e^{2i\theta}) = -i(-1 + e^{2i\theta}) \\
	x + xe^{2i\theta} = i - ie^{2i\theta} \\
	xe^{2i\theta} + ie^{2i\theta} = i - x \\
	e^{2i\theta}(i + x) = i - x \\
	e^{2i\theta} = \frac{i - x}{i + x} \\
	2i\theta = \ln(\frac{i - x}{i + x}) \\
	i\theta = \frac{1}{2}\ln(\frac{i - x}{i + x}) \\
	\theta = -\frac{i}{2}\ln(\frac{i - x}{i + x})
\end{gather*}

\begin{gather*}
	\therefore \arctan(\theta) = -\frac{i}{2}\ln(\frac{i - \theta}{i + \theta})
\end{gather*}
\end{document}
